\documentclass[a4paper,10pt]{article}
\input{/Users/benjamin/Documents/Education/LaTeX/macro.tex}

\title{ASE: S�ance 1}
\author{Fran�ois \bsc{Lepan}\\Benjamin \bsc{Van Ryseghem}}

\begin{document}
\maketitle

\section{Retour � un contexte}

\verb?#include <setjmp.h>?

\verb?int setjmp(jmp_buf ctx)? save the current context into the \verb?ctx? variable.
\begin{itemize}
	\item[] first time, returns 0;
	\item[] second time returns something different from 0.
\end{itemize}

\verb?void longjmp(jmp_buf ctx, int val)? where val is different from 0.
Resume the cx context.

\begin{attention}
Beware of not keeping a context to a dead memory space (same principle that referring to a freed pointer)
\end{attention}


\section{Exercice 2}
Ce programme est erron� car le \verb?longjmp? change le contexte courant pour un contexte appartenant la partie de la pile qui a d�j� �t� lib�r�e.


\section{Exercice 3}
\begin{verbatimtab}
jmp_buf buffer;
static int mul(int depth) {
	int i;
	switch (scanf("%d", &i)) { 
		case EOF :
			return 1; /* neutral element */
		case 0:
			return mul(depth+1); /* erroneous read */
		case 1 :
			if (i)
				return i * mul(depth+1);	
			else
				return longjmp(buffer, ~0);
	}
}
				
int main() {
	int product;
	printf("A list of int, please\n"); 
	if (!setjmp(buffer)){
		product = mul(0);
	}
	else { product = 0; } */
	printf("product = %d\n", product);
}
\end{verbatimtab}

\signature[Fran�ois \bsc{Lepan}\\Benjamin \bsc{Van Ryseghem}]

\end{document}