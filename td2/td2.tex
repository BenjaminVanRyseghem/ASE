\documentclass[a4paper,10pt]{article}
\input{/Users/benjamin/Documents/Education/LaTeX/macro.tex}

\title{ASE: S�ance 2}
\author{Fran�ois \bsc{Lepan}\\Benjamin \bsc{Van Ryseghem}}

\begin{document}
\maketitle

\part*{Cr�ation d'un contexte d'ex�cution}
\section{Exercice 5}
\begin{verbatimtab}
typedef void (func_t) (void*);
typedef enum {CTX_RDY, CTX_EXQ, CTX_END} state_e;

struct ctx_s {	
	void* ctx_esp;
	void* ctx_ebp;
	unsigned ctx_magic;
	func_t* ctx_f;
	void* ctx_arg;
	enum state_e ctx_state;
	char* ctx_stack; /* adresse la plus basse de la pile */
	unsigned ctx_size;
};
\end{verbatimtab}

\section{Exercice 6}
\begin{verbatimtab}
int init_ctx(struct ctx_s *ctx, int stack_size, func_t f, void *args){
	ctx->ctx_stack = (char*) malloc(stack_size);
	if ( ctx->ctx_stack == NULL) return 1;
	ctx->ctx_state = CTX_RDY;
	ctx->ctx_size = stack_size;
	ctx->ctx_f = f;
	ctx->ctx_arg = args;
	ctx->ctx_esp = &(ctx->ctx_stack[stack_size-sizeof(int)]);
	ctx->ctx_ebp = ctx->ctx_esp;
	return 0;
}
\end{verbatimtab}

\signature[Fran�ois \bsc{Lepan}\\Benjamin \bsc{Van Ryseghem}]

\end{document}